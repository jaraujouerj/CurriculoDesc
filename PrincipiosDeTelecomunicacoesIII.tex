%----------------------------------------------------------------------------------------
%	Latex para ementa tipo dep da uerj
%----------------------------------------------------------------------------------------
\documentclass{ementa} % Use A4 paper with a 12pt font size 

\begin{document} 
	\ndisciplina 	{\PrincTelec}
	\chta 				{\PrincTelecCH h}%carga horária total aluno
	\codigo				{FEN 05-04975 }
	\alteracao
	\departamento		{Engenharia  Eltrônica e de Telecomunicações}
	\preum				{\CEVI}
	\codpreum			{\CEVICod} %
	\credteorica		{3} %
	\credpratica		{1} %
	\objetivos			{Introduzir os princípios básicos da teoria de telecomunicações em termos da análise de freqüência dos
sinais e dos ruídos, e dos diferentes sistemas de modulação, codificação, processamento da informação
e da mensagem.
	}
	\ementa				{Sinais eletromagnéticos da informação ou mensagem, e sua representação no domínio do tempo e da
freqüência. Elementos da teoria da amostragem e suas aproximações. Transmissão dos sinais através
de sistemas lineares. Características dos filtros dos sistemas de transmissão. Transmissão sem
distorções. Filtros ideais. Realizabilidade física e causalidade. Relação entre a largura de faixa e o rise-
time. Características de tempo de resposta de pulsos. Espectros de freqüência de energia e potência.
Sinais de energia e sinais de potência. Sinais de modulação: AM, FM e modulação por pulsos (PCM).
Multiplexação por divisão de freqüência e por divisão de tempo.
	}
	\formementa{PrincTelec}
\end{document}
