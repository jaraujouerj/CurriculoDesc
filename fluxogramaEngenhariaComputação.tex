\documentclass[a4paper, landscape]{article}
\usepackage{prerex}
\usepackage{multicol}
%\usepackage{showframe}
\usetikzlibrary{calc}
\usepackage[utf8]{inputenc} % codificacao de caracteres
\usepackage{geometry}
\usepackage[portuguese]{babel}
\usepackage{tikz}
\tikzset{
    %Define standard arrow tip
    >=stealth,
    %Define style for boxes
    box/.style={
           rectangle,
           draw=black,
           text width=7em,
           minimum height=5em,
           text centered,
           inner sep=1mm]},
    % Define arrow style
    pil/.style={
           ->,
           thick,
           shorten <=3pt,
           shorten >=3pt,},
}
\usetikzlibrary{positioning,shapes,shadows,arrows}
\tikzstyle{basico}=[box, fill=yellow]
\tikzstyle{desc}=[box, fill=orange]
\tikzstyle{profissional}=[box, fill=green]
\tikzstyle{eletiva}=[box, fill=orange!50!white, dashed]
\newcounter{cred}
\setcounter{cred}{0}
\newcounter{thoras}
\setcounter{thoras}{0}
\newcommand{\disciplina}[8] {
  \addtocounter{cred}{#7}; \addtocounter{thoras}{#6};
  \node at ($ (#2*2.8-3,15-#3*2.3) $) (Item)  [#1] (#8) 
        {
            \textbf{#4}\\
            \textbf{#5}\\
            \textbf{#6} \hfill #7
        };}
\newcommand{\prereq}[2] {
  \draw [->, thick](#1.east)--(#2.west);}
%nó acima
\newcommand{\prereqC}[2] {
   \draw [->, thick]
    ($(#1.east) + (0,2mm) $) %comece 2mm abaixo de east 
    -| ($ (#2.west)+(-3mm,-2mm)$) %quebre 90º até chegar a 3mm ao lado e 2 abaixo de west 
    -- ($(#2.west)-(0,2mm)$); %termine 2 mm abaixo de west
    }
\newcommand{\prereqD}[2] {
   \draw [->, thick]
    ($(#1.east) + (0,-2mm) $) %comece 2mm abaixo de east  
    -| ($ (#2.west) -(3mm,-2mm)$) %quebre 90º até chegar a 3mm ao lado e 2 abaixo de west
    -- ($(#2.west) + (0,2mm)$); termine 2 mm acima de west
    }
\newcommand{\prereqE}[2] {
   \draw [->, thick]
    ($(#1.east) + (0,-2mm) $) -| ($(#1.east) + (3mm,-10mm) $) -| ($ (#2.west) -(4mm,-2mm)$) -- ($(#2.west) + (0,2mm)    $);}
\newcommand{\prereqJ}[2] {
   \draw [->, thick]
    ($(#1.east) $) -| ($(#1.east) + (1mm,-10mm) $) -| ($ (#2.west) -(4mm,-2mm)$) -- ($(#2.west) + (0,2mm)    $);}
       
\newcommand{\prereqH}[2] {
       \draw [->, thick]
        ($(#1.east) + (0,-2mm) $) -| ($(#1.east) + (3mm,-10mm) $) -| ($ (#2.west) -(3mm,-2mm)$) -- ($(#2.west) + (0,2mm)    $);}
\newcommand{\prereqI}[2] {
       \draw [->, thick]
        ($(#1.east) $) -| ($(#1.east) + (4mm,-9mm) $) -| ($ (#2.west) -(3mm,-2mm)$) -- ($(#2.west) + (0,2mm)    $);}
\newcommand{\prereqA}[2] {
   \draw [->, thick]
    ($(#1.east) + (0,2mm) $) -| ($(#1.east) + (1mm,10mm) $) -| ($ (#2.west) -(4mm,0)$) -- ($(#2.west)    $);}
\newcommand{\prereqB}[2] {
  \draw [->, thick]
    ($(#1.east) + (0,-1mm) $)  -| ($ (#2.west) -(2mm,-4mm)$) -- ($(#2.west) + (0,4mm)    $);}
\geometry{margin=1cm,bottom=1.5cm} 

\newcommand{\prereqF}[2] {
   \draw [->, thick]
    ($(#1.east) + (0,2mm) $) -| ($(#1.east) + (1mm,10mm) $) -| ($ (#2.west) -(3mm,0)$) -- ($(#2.west)    $);}

\newcommand{\prereqG}[2] {
   \draw [->, thick]
    ($(#1.east) + (0,2mm) $) -| ($(#1.east) + (1mm,10mm) $) -| ($ (#2.west) -(3mm,2mm)$) -- ($(#2.west) -(0,2mm)   $);}
    
\newcommand{\prereqK}[2] {
   \draw [->, thick]
    ($(#1.east) + (0,2mm) $) -| ($(#1.east) + (1mm,12mm) $) -| ($ (#2.west) -(4mm,0)$) -- ($(#2.west)    $);}
\newcommand{\prereqL}[2] {
   \draw [->, thick]
    ($(#1.east) + (0,-3mm) $) -| ($(#1.east) + (1mm,-35mm) $) -| ($ (#2.west) -(4mm,-2mm)$) -- ($(#2.west) + (0,2mm)    $);}
\newcommand{\prereqM}[2] {
   \draw [->, thick]
    ($(#1.east) + (0,-2mm) $) -| ($(#1.east) + (3mm,-9mm) $) -| ($ (#2.west) -(4mm,-2mm)$) -- ($(#2.west) + (0,2mm)    $);} 
\newcommand{\prereqN}[2] {
   \draw [->, thick]
    ($(#1.east) + (0,2mm) $) -| ($(#1.east) + (1mm,11mm) $) -| ($ (#2.west) -(3mm,3mm)$) -- ($(#2.west) -(0,3mm)   $);}
\newcommand{\prereqO}[2] {
       \draw [->, thick]
        ($(#1.east) + (0,-2mm) $) -| ($(#1.east) + (1mm,-14mm) $) -| ($ (#2.west) -(3mm,0mm)$) -- ($(#2.west)    $);}  
\newcommand{\prereqP}[2] {
       \draw [->, thick]
        ($(#1.east) + (0,-2mm) $) -| ($(#1.east) + (3mm,-12mm) $) -| ($ (#2.west) -(2mm,-2mm)$) -- ($(#2.west) + (0,2mm)    $);} 
\newcommand{\prereqQ}[2] {
        \draw [->, thick]
         ($(#1.east) + (0,-2mm) $) -| ($(#1.east) + (3mm,-11mm) $) -| ($ (#2.west) -(3mm,-2mm)$) -- ($(#2.west) + (0,2mm)    $);}
\begin{document}
\noindent
{\Large \textbf{Curso: Engenharia}}\\
{\Large \textbf{Habilitação: Sistemas e Computação}}  (Versão de \today)

\begin{tikzpicture}
\foreach \x in {1,...,10} {\node at (\x*2.8-3,13.7) {\xº Período};}

\footnotesize 
\fontfamily{anttlc}
%parâmetros de cada disciplina, pode ser posto numa única linha
%\disciplina
%{basico || profissional || desc || eletiva}
%{Período} de 1 a 10
%{Linha no fluxograma} de 1 a 6.
%{Código da disciplina}
%{Nome da Disciplina}
%{Carga Horária}
%{Número de Créditos}
%{nome abreviado para referência}

%Básico
\disciplina{basico}{2}{6}{IME 02-01388}	{Álgebra\\ Linear III}{75}{5}{AlgLin}
\disciplina{basico}{3}{2}{IME 02-04629}	{Análise\\ Vetorial}{60}{4}{AnaVet}
\disciplina{basico}{1}{3}{IME 01-00508}	{Cálculo Dif. e Integral I}{75}{5}{CalcI}
\disciplina{basico}{2}{3}{IME 01-00854}	{Cálculo Dif. e Integral II}{75}{5}{CalcII}
\disciplina{basico}{3}{3}{IME 01-03646}	{Cálculo Dif. e Integral III}{75}{5}{CalcIII}
\disciplina{basico}{2}{1}{IME 04-04541}	{Cálculo Numérico IV}{60}{4}{CalcNum}
\disciplina{basico}{2}{4}{IME 03-00587}	{Desenho\\ Básico}{60}{3}{DesBas}
\disciplina{basico}{1}{5}{FIS 01-05095}	{Física Teórica e\\Experimental I}{105}{5}{FisI}
\disciplina{basico}{2}{5}{FIS 02-05143}	{Física Teórica e\\Experimental II}{105}{5}{FisII}
\disciplina{basico}{3}{5}{FIS 03-05185}	{Física Teórica e\\Experim. III}{120}{6}{FisIII}
\disciplina{basico}{4}{5}{FIS 04-05212}	{Física Teórica e\\Experim. IV}{120}{6}{FisIV}
\disciplina{basico}{1}{4}{IME 03-02046}	{Geometria Descritiva I}{60}{3}{GD}
\disciplina{basico}{1}{6}{IME 03-01913}	{Geom. Analít.\\ e Cálc. Vet. I}{75}{5}{GeoAna}
\disciplina{basico}{3}{4}{FEN 03-05787}	{Mecânica\\ Técnica}{60}{4}{MecTec}
\disciplina{basico}{3}{6}{IME 05-05316}	{Probabilidade e\\ Estatística III}{75}{5}{ProbEst}
\disciplina{basico}{1}{2}{QUI 07-03793}	{Química\\ X}{60}{3}{QuiX}
\disciplina{basico}{2}{2}{QUI 07-03865}	{Química\\ XI}{60}{3}{QuiXI}

%disciplinas profissional
\disciplina{profissional}{10}{6}{FAF 03-04439}	{Administração Aplic. a Eng. III}{60}{3}{Adm}
\disciplina{profissional}{6}{4}	{FEN 04-05253}	{Análise de Sistemas Físicos I}{75}{4}{AnaFis}
\disciplina{profissional}{6}{6}	{FEN 05-00662}	{Circuitos Combinacionais e Sequenciais}{60}{3}{CirComb}
\disciplina{profissional}{4}{6}	{FEN 04-00944}	{Circuitos Elétricos I}{90}{4}{CEI}
\disciplina{profissional}{5}{6}	{FEN 04-01129}	{Circuitos Elétricos IV}{90}{4}{CEIV}
\disciplina{profissional}{7}{4}	{FEN 05-05025}	{Controle e Servomecanismos III}{75}{4}{ServoMec}
\disciplina{profissional}{7}{5}	{FEN 04-05153}	{Conv. Eletrom. de Energia III}{90}{4}{ConvEle}
\disciplina{profissional}{5}{5}	{FEN 05-01620}	{Eletrônica\\ I}{90}{4}{EletI}
\disciplina{profissional}{6}{5}	{FEN 05-01840}	{Eletrônica\\ II}{90}{4}{EletII}
\disciplina{profissional}{4}{3}	{FEN 03-02040}	{Fenômenos de Transporte}{60}{3}{FenTran}
\disciplina{profissional}{9}{5}	{FCE 02-04657}	{Introdução à Economia III}{60}{4}{IntEco}
\disciplina{profissional}{7}{6}	{FEN 07-02162}	{Introdução à Engenharia Ambiental}{60}{3}{IntAmb}
\disciplina{profissional}{4}{2}	{FEN 04-05197}	{Mat. Elét. e Magnéticos I}{60}{3}{MatEle}
\disciplina{profissional}{5}{4} {FEN 05-04923}	{Modelos Matemáticos III}{75}{4}{ModMat}
\disciplina{profissional}{8}{6} {FEN 09-7345}	{Métodos Quant. Aplic. em Produção I}{45}{3}{PO}
\disciplina{profissional}{7}{3}	{FEN 05-04975}	{Princípios de Telec. III}{75}{4}{PrincTelec}
\disciplina{profissional}{4}{4}	{FEN 01-04833}	{Resistência dos Mat. Básica}{60}{3}{ResMat}
\disciplina{profissional}{9}{6}	{FEN 07-02722}	{Segurança e Higiene no Trabalho}{30}{2}{SegHig}

%Disciplinas Desc
\disciplina{desc}{1}{1} {FEN 06-xxxx}	{Algoritmos\\Computacionais}{75}{4}{AlgComp}
\disciplina{desc}{6}{2} {FEN 06-xxxx}	{Análise de Algoritmos}{60}{3}{AnAlg}
\disciplina{desc}{7}{2} {FEN 06-xxxx}	{Arquitetura de Computadores}{75}{4}{ArqComp}
\disciplina{desc}{8}{2} {FEN 06-xxxx}	{Arquitetura de Sistemas Operacionais}{75}{5}{SO}
\disciplina{desc}{8}{1} {FEN 06-xxxx}	{Compiladores}{75}{3}{Comp}
\disciplina{desc}{10}{3}{FEN 06-xxxx}	{Cont. de Proc. por Comput.}{75}{4}{Control}
\disciplina{desc}{9}{1} {FEN 06-xxxx}	{Eng., Tec. e Sociedade}{60}{3}{ETS}
\disciplina{desc}{6}{1} {FEN 06-xxxx}	{Engenharia de Sistemas A}{60}{3}{EngSistA}
\disciplina{desc}{7}{1} {FEN 06-xxxx}	{Engenharia de Sistemas B}{60}{3}{EngSistB}
\disciplina{desc}{9}{3} {FEN 06-xxxx}	{Estágio Supervisionado}{30}{1}{EstSup}
\disciplina{desc}{3}{1} {FEN 06-xxxx}	{Estrutura da Informação}{60}{3}{EstrInf}
\disciplina{desc}{5}{3} {FEN 06-xxxx}	{Fundamentos de Computadores}{60}{3}{FundComp}
\disciplina{desc}{6}{3} {FEN 06-xxxx}	{Inteligência Computacional}{60}{3}{IC}
\disciplina{desc}{4}{1} {FEN 06-xxxx}	{Laboratório de Programação I}{60}{3}{LabProgI}
\disciplina{desc}{5}{1} {FEN 06-xxxx}	{Laboratório de Programação II}{60}{3}{LabProgII}
\disciplina{desc}{5}{2} {FEN 06-xxxx}	{Lógica em Programação}{60}{3}{LogProg}
\disciplina{desc}{8}{5} {FEN 06-xxxx}	{Processamento de Imagens}{60}{3}{ProcImag}
\disciplina{desc}{9}{2} {FEN 06-xxxx}	{Programação Paralela}{60}{3}{ProgPara}
\disciplina{desc}{9}{4} {FEN 06-xxxx}	{Projeto de Graduação A}{45}{2}{ProjA}
\disciplina{desc}{10}{4}{FEN 06-xxxx}	{Projeto de Graduação B}{45}{2}{ProjB}
\disciplina{desc}{8}{4} {FEN 06-xxxx}	{Sistemas Embutidos}{60}{3}{SistEmb}
\disciplina{desc}{8}{3} {FEN 06-xxxx}	{Telep. e Redes de Comput.}{60}{3}{Telep}

%eletivas
\disciplina{eletiva}{9}{7} {FEN 06-xxxx}	{Eletiva A}{60}{3}{EA}
\disciplina{eletiva}{10}{1}{FEN 06-xxxx}	{Eletiva B}{60}{3}{EB}
\disciplina{eletiva}{10}{2}{FEN 06-xxxx}	{Eletiva C}{60}{3}{EC}
\disciplina{eletiva}{10}{5}{FEN 06-xxxx}	{Eletiva D}{60}{3}{ED}

\prereq				{ModMat}	{AnaFis}
\prereqC			{CalcII}	{AnaVet}
\prereq				{CalcI}		{CalcII}
\prereqC			{CalcI}		{CalcNum}
\prereq				{AlgComp}	{CalcNum}
\prereq				{CalcII}	{CalcIII}
\prereq				{EletII}	{CirComb}%problema
\prereq				{CEI}		{CEIV}
\prereq				{ModMat}	{CEIV} %problema
\prereq				{GD}		{DesBas}
\prereqM			{FisIII}	{EletI}
\prereq				{EletI}		{EletII}
\prereqG			{CEI}		{EletII}
\prereq				{EngSistA}	{EngSistB}
\prereqQ			{EstrInf}	{EngSistB}
\prereqA			{FisII}		{FenTran}
\prereqD			{AnaVet}	{FenTran}
\prereq				{FisI}		{FisII}
\prereqD			{CalcI}		{FisII}
\prereq				{FisII}		{FisIII}
\prereq				{FisIII}	{FisIV}
\prereqA			{FisI}		{MecTec}
\prereqB			{CalcII}	{MecTec}
\prereqF			{CalcIII}	{ModMat}
\prereq				{QuiX}		{QuiXI}
\prereqE			{AlgLin}	{PO}
\prereqD			{CalcII}	{ProbEst}
\prereq				{ProjA}		{ProjB}
\prereq				{MecTec}	{ResMat}
\prereq				{AnaFis}	{ServoMec}
%
\prereqO			{AlgComp}	{AnAlg}
\prereqP			{LabProgI}	{AnAlg}
\prereqA			{AnAlg}		{Comp}
\prereqE			{SO}		{Control}
\prereqA			{AnaFis}	{Control}
\prereqI			{LabProgII}	{ETS}
\prereqL			{LabProgI}	{IC}
\prereqL			{LabProgII}	{ProcImag}
\prereqK			{AlgLin}	{ProcImag}
\prereqN			{ProbEst}	{ProcImag}
\prereq				{SO}		{ProgPara}
\prereq				{ArqComp}	{SO}
\prereqH			{LabProgII}	{SO}
\prereqB			{ArqComp}	{Telep}
\prereq				{PrincTelec}{Telep}



%\node [above of=IntAmb] 
%        {
%             \textbf{110 cr.}\\
%        };

%\filldraw[color=yellow, draw=black] rectangle (0.3,0.3) at (\textwidth/10,15-7*2.3);
\normalsize 
\node at (0.9cm,-2mm) (legenda){Código para as Disciplinas:};
\node at (-0.9cm,-7mm) [fill=yellow, draw=black, label=right:Básico da Engenharia.] (basico) {};
\node [below  of=basico,node distance=5mm,fill=green, draw=black, label=right:Profissional Comum.] (prof) {};
\node [below of=prof,node distance=5mm,fill=orange, draw=black, label=right:Profissional Específico.] (especifico){};
\node [below of=especifico,node distance=5mm,fill=orange!50!white, draw=gray, draw=black, label=right:Eletivas] {};

\node at (6cm,-1.05) [ text width=4.6cm] {Os números situados na parte inferior direita e esquerda representam os créditos e carga horária, respectivamente, conferidos à disciplina.};

\node [below left=-5mm of SegHig,text width=20pt, font=\scriptsize] (110cred) {110 cred.};
\newcommand{\prereqCr}[2] {
  \draw [->, thick]
    ($(#1.north) - (1mm,0) $)  -| ($ (#2.west) -(2mm,0mm)$) -- ($(#2.west)   $);}
\prereqCr {110cred}{SegHig}

%tirando duas eletiva
\addtocounter{thoras}{-120} 
\addtocounter{cred}{-6}


\node at (11.5cm,-1) [ text width=5cm]
{O Curso de Engenharia será integralizado em um mínimo de 10 e um máximo de 18 períodos compreendendo: Parte Comum e Ciclo Profissional.};

\node at (17,-0.8) [ text width=5cm] {Total de Créditos do curso: \the\value{cred}\\
Total de horas do curso: \the\value{thoras}} ;


\end{tikzpicture}
\date{\currenttime}
%Fisica III pré de Circuitos I?
%Circuitos Elétricos IV?
%Modelos Matem. III pré de Circuitos II (IV)?
%Circuitos II pré de Principios de Teleco?
%Algebra Linear pré de Conversão Eletrom. (é pré de eletricicdade II)
%Circuitos Combinacionais e Sequencias tem pré letrônica II, mas estão sendo dados no mesmo período.
%Segurança e Higiene no Trabalho pré é 110 créditos.

\end{document}
