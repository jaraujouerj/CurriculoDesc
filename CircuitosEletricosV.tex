%----------------------------------------------------------------------------------------
%	Latex para ementa tipo dep da uerj
%----------------------------------------------------------------------------------------
\documentclass{ementa} % Use A4 paper with a 12pt font size 

\begin{document}
	\ndisciplina 		{\CEV}
	\chta 				{\CEVCH h}%carga horária total aluno
	\codigo				{\CEVCod}
	\alteracao
	\departamento		{Engenharia  Elétrica}
	\preum				{\FisIII}
	\codpreum			{\FisIIICod} %
	\credteorica		{2} %
	\credlab			{1}
	\credpratica		{1} %
	\objetivos	{Ao  final  do  período  o  aluno  deverá  ser  capaz  de  analisar  circuitos  elétricos  lineares  a parâmetros concentrados no domínio do tempo. 
	}
	\ementa	{Conceitos básicos. Elementos passivos e ativos. Leis de Kirchoff. Linearidade. Teoremas de Thevenin e  Norton. Topologia dos circuitos.  Métodos  das  malhas  e  dos  nós.  Funções/Singulares. Respostas  no tempo de circuitos de 1ª e 2ª ordem. Convolução. 
	}
	\formementa{circuitosV}
\end{document}
