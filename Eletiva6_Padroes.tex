%----------------------------------------------------------------------------------------
%	Latex para ementa tipo dep da uerj
%----------------------------------------------------------------------------------------
%apagar .aux e executar biber ementalatex para obter bibliografia atualizada
\documentclass{ementa} % Use A4 paper with a 12pt font size 

\begin{document}
\ndisciplina 		{\EletPadroes}
\nomeprofessor		{Oscar Luiz Monteiro de Farias}
\matriculaprofessor	{31998-8}
\chta 				{\EletPadroesCH h}%carga horária total aluno
\naoobrigatoria		%disciplina não é obrigatória
\eletivarestrita	{}
\preum				{Laboratório de Programação B} %
\codpreum			{FEN06-xxxxx} %
\travadecreditos   	{170} % trava de créditos
\credteorica		{2} %
\credlab			{1} %
\objetivos			{O objetivo desse curso é familiarizar os alunos com técnicas 
					voltadas para o projeto de sistemas orientados a objetos, enfatizando a utilização de padrões.  Espera-se que ao término do curso os alunos estejam aptos a compreender os padrões de projetos mais relevantes e a empregá-los no projeto de sistemas orientados a objetos que sejam flexíveis, reusáveis, de fácil alteração e manutenção e que possam evoluir graciosamente ao longo do tempo.}
\ementa				{Princípios utilizados no projeto de sistemas orientados a objetos. Técnicas que possibilitam o reuso de componentes 
					modulares de software, organizados de forma cooperativa e que facilitam a manutenção e evolução dos sistemas orientados a objetos. Padrões de Projetos de Sistemas Orientados a Objetos (design patterns for object-oriented software systems).}
\formementa{Eletiva6_PadroesdeProj}
\end{document}
