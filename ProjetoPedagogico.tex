%%%%%%%%%%%%%%%%%%%% book.tex %%%%%%%%%%%%%%%%%%%%%%%%%%%%%
%
% sample root file for the chapters of your "monograph"
%
% Use this file as a template for your own input.
%
%%%%%%%%%%%%%%%% Springer-Verlag %%%%%%%%%%%%%%%%%%%%%%%%%%


% RECOMMENDED %%%%%%%%%%%%%%%%%%%%%%%%%%%%%%%%%%%%%%%%%%%%%%%%%%%
\documentclass[envcountsame,envcountchap,openany]{svmono}
%\documentclass[11pt]{book} 
% teste
%\usepackage[dvinames]{xcolor}
% Measurements are taken directly from the guide
\usepackage[top=1in,left=1.5in,bottom=1in,right=2.5cm]{geometry}
%\usepackage{graphicx}
%\usepackage[colorlinks=false,
%            pdfborder={0 0 0},
%            ]{hyperref}
%\usepackage{lipsum}
\usepackage[absolute]{textpos}
\usepackage[table]{xcolor}
\usepackage{tikz}
\usetikzlibrary{calc}
\usepackage{scrextend}
\usepackage{tabularx}
\usepackage{spreadtab}

% A nice serif font, but no the prescribed nonfree ITC stone
%\usepackage[oldstylenums]{kpfonts}
\usepackage[T1]{fontenc}
\usepackage[utf8]{inputenc}
\usepackage[portuguese]{babel}
\usepackage{enumerate}
\usepackage{multirow}
\usepackage[titletoc,toc,page]{appendix}
\usepackage{pdfpages}
%\usepackage[titletoc]{appendix}
%\renewcommand{\appendixtocname}{Anexos}
\renewcommand{\appendixtocname}{Anexos}
\renewcommand{\appendixpagename}{Anexos}

\newcounter{artigo}
\newcommand{\artigo}{\refstepcounter{artigo} % 
	\ifnum\theartigo<10 %
	{\bfseries Art.~\arabic{artigo}º~--~}%
	\else
	{\bfseries Art. \arabic{artigo}~--~}%
	\fi
	%Art. \arabic{artigo}.~
	}
\newcounter{paragrafo}
\newcommand{\paragrafo}{\refstepcounter{paragrafo} % 
	\S~\arabic{paragrafo}º~%
}

\newenvironment{paragrafos}{\setcounter{paragrafo}{0}
	\setlength{\parindent}{0pt}
	\begin{addmargin}[4em]{0pt} 
	}
	{\end{addmargin}
		}

\newenvironment{itquotation}
{\begin{quotation}\itshape}
{\end{quotation}}	
% No paragraph indentation
\parindent0pt
\setlength{\parskip}{0.8\baselineskip}
%\raggedright
%\pagestyle{empty}

%fim teste
% choose options for [] as required from the list
% in the Reference Guide, Sect. 2.2

\usepackage{makeidx}         % allows index generation
\usepackage{graphicx}        % standard LaTeX graphics tool
                             % when including figure files
\usepackage{multicol}        % used for the two-column index
\usepackage[bottom]{footmisc}% places footnotes at page bottom

% etc.
% see the list of further useful packages
% in the Reference Guide, Sects. 2.3, 3.1-3.3

\makeindex             % used for the subject index
                       % please use the style svind.ist with
                       % your makeindex program

% Nomes das Disciplinas

% Nomes das Disciplinas %%%%%%%%%%%%%%%%%%%%%%%%%%%%%%%%%%%%%%%%%%%%%%%%%%%%%%%%%%%%%%%%%%%%
\newcommand{\AlgLin}{Álgebra Linear III}
\newcommand{\AlgLinSName}{Álgebra Linear III}
\newcommand{\AlgLinCod}{IME 02-01388}
\newcommand{\AlgLinCH}{75}
\newcommand{\AlgLinCred}{5}

\newcommand{\AnaVet}{Análise Vetorial}
\newcommand{\AnaVetSName}{Análise Vetorial}
\newcommand{\AnaVetCod}{IME 02-04629}
\newcommand{\AnaVetCH}{60}
\newcommand{\AnaVetCred}{4}

\newcommand{\CalcI}{Cálculo Diferencial e Integral I}
\newcommand{\CalcISName}{Cálculo Dif. e Integral I}
\newcommand{\CalcICod}{IME 01-00508}
\newcommand{\CalcICH}{75}
\newcommand{\CalcICred}{5}

\newcommand{\CalcII}{Cálculo Diferencial e Integral II}
\newcommand{\CalcIISName}{Cálculo Dif. e Integral II}
\newcommand{\CalcIICod}{IME 01-00854}
\newcommand{\CalcIICH}{75}
\newcommand{\CalcIICred}{5}

\newcommand{\CalcIII}{Cálculo Diferencial e Integral III}
\newcommand{\CalcIIISName}{Cálculo Dif. e Integral III}
\newcommand{\CalcIIICod}{IME 01-03646}
\newcommand{\CalcIIICH}{75}
\newcommand{\CalcIIICred}{5}

\newcommand{\DesBas}{Desenho Básico}
\newcommand{\DesBasSName}{Desenho Básico}
\newcommand{\DesBasCod}{IME 03-00587}
\newcommand{\DesBasCH}{60}
\newcommand{\DesBasCred}{3}

\newcommand{\FisI}{Física Teórica e Experimental I}
\newcommand{\FisISName}{Física Teórica e Experim. I}
\newcommand{\FisICod}{FIS 01-05095}
\newcommand{\FisICH}{105}
\newcommand{\FisICred}{5}

\newcommand{\FisII}{Física Teórica e Experimental II}
\newcommand{\FisIISName}{Física Teórica e Experim. II}
\newcommand{\FisIICod}{FIS 02-05143}
\newcommand{\FisIICH}{105}
\newcommand{\FisIICred}{5}

\newcommand{\FisIII}{Física Teórica e Experimental III}
\newcommand{\FisIIISName}{Física Teórica e Experim. III}
\newcommand{\FisIIICod}{FIS 03-05185}
\newcommand{\FisIIICH}{120}
\newcommand{\FisIIICred}{6}

\newcommand{\FisIV}{Física Teórica e Experimental IV}
\newcommand{\FisIVSName}{Física Teórica e Experim. IV}
\newcommand{\FisIVCod}{FIS 04-05212}
\newcommand{\FisIVCH}{120}
\newcommand{\FisIVCred}{6}

\newcommand{\GD}{Geometria Descritiva I}
\newcommand{\GDSName}{Geometria Descritiva I}
\newcommand{\GDCod}{IME 03-02046}
\newcommand{\GDCH}{60}
\newcommand{\GDCred}{3}

\newcommand{\GeoAna}{Geometria Analítica e Cálculo Vetorial I}
\newcommand{\GeoAnaSName}{Geom. Analít. e Cálc. Vet. I}
\newcommand{\GeoAnaCod}{IME 03-01913}
\newcommand{\GeoAnaCH}{75}
\newcommand{\GeoAnaCred}{5}

\newcommand{\MecTec}{Mecânica Técnica}
\newcommand{\MecTecSName}{Mecânica Técnica}
\newcommand{\MecTecCod}{FEN 03-05787}
\newcommand{\MecTecCH}{60}
\newcommand{\MecTecCred}{4}

\newcommand{\ProbEst}{Probabilidade e Estatística III}
\newcommand{\ProbEstSName}{Probabilidade e Estatística III}
\newcommand{\ProbEstCod}{IME 05-05316}
\newcommand{\ProbEstCH}{75}
\newcommand{\ProbEstCred}{5}

\newcommand{\QuiX}{Química X}
\newcommand{\QuiXSName}{Química X}
\newcommand{\QuiXCod}{QUI07-03793}
\newcommand{\QuiXCH}{60}
\newcommand{\QuiXCred}{3}

\newcommand{\Adm}{Administração Aplicada à Engenharia III}
\newcommand{\AdmSName}{Adm. Aplic. à Engenharia III}
\newcommand{\AdmCod}{FAF 03-04439}
\newcommand{\AdmCH}{60}
\newcommand{\AdmCred}{3}

\newcommand{\AnaFis}{Análise de Sistemas Físicos I-A}
\newcommand{\AnaFisSName}{Análise de Sistemas Físicos I-A}
\newcommand{\AnaFisCod}{FEN 04-xxxxx}
\newcommand{\AnaFisCH}{75}
\newcommand{\AnaFisCred}{4}

\newcommand{\CEV}{Circuitos Elétricos V}
\newcommand{\CEVSName}{Circuitos Elétricos V}
\newcommand{\CEVCod}{FEN 04-xxxxx}
\newcommand{\CEVCH}{90}
\newcommand{\CEVCred}{4}

\newcommand{\CEVI}{Circuitos Elétricos VI}
\newcommand{\CEVISName}{Circuitos Elétricos VI}
\newcommand{\CEVICod}{FEN 04-xxxxx}
\newcommand{\CEVICH}{90}
\newcommand{\CEVICred}{4}

\newcommand{\CServMec}{Controle e Servomecanismos III-A}
\newcommand{\CServMecSName}{Controle e Servomecanismos III-A}
\newcommand{\CServMecCod}{FEN 05-xxxxx}
\newcommand{\CServMecCH}{75}
\newcommand{\CServMecCred}{4}

\newcommand{\EletI}{Eletrônica I}
\newcommand{\EletISName}{Eletrônica I}
\newcommand{\EletICod}{FEN 05-01620}
\newcommand{\EletICH}{90}
\newcommand{\EletICred}{4}

\newcommand{\EletIIA}{Eletrônica II-A}
\newcommand{\EletIIASName}{Eletrônica II-A}
\newcommand{\EletIIACod}{FEN 05-xxxx}
\newcommand{\EletIIACH}{90}
\newcommand{\EletIIACred}{4}

\newcommand{\FenTran}{Fenômenos de Transporte}
\newcommand{\FenTranSName}{Fenômenos de Transporte}
\newcommand{\FenTranCod}{FEN 03-02040}
\newcommand{\FenTranCH}{60}
\newcommand{\FenTranCred}{3}

\newcommand{\IntEco}{Introdução à Economia III}
\newcommand{\IntEcoSName}{Introdução à Economia III}
\newcommand{\IntEcoCod}{FCE 02-04657}
\newcommand{\IntEcoCH}{60}
\newcommand{\IntEcoCred}{4}

\newcommand{\IntAmb}{Introdução à Engenharia Ambiental}
\newcommand{\IntAmbSName}{Introd. à Eng. Ambiental}
\newcommand{\IntAmbCod}{FEN 07-02162}
\newcommand{\IntAmbCH}{60}
\newcommand{\IntAmbCred}{3}

\newcommand{\MatEle}{Materiais Elétricos e Magnéticos I-A}
\newcommand{\MatEleSName}{Mat. Elét. e Magnéticos I-A}
\newcommand{\MatEleCod}{FEN 04-xxxxx}
\newcommand{\MatEleCH}{60}
\newcommand{\MatEleCred}{3}

\newcommand{\ModMat}{Modelos Matemáticos Aplicados à Engenharia Elétrica III}
\newcommand{\ModMatSName}{Modelos Matemáticos III}
\newcommand{\ModMatCod}{FEN 05-04923}
\newcommand{\ModMatCH}{75}
\newcommand{\ModMatCred}{4}

\newcommand{\MetQuant}{Métodos Quantitativos Aplicados em Produção I}
\newcommand{\MetQuantSName}{Métodos Quant. Aplic. em Produção I}
\newcommand{\MetQuantCod}{FEN 09-7345}
\newcommand{\MetQuantCH}{45}
\newcommand{\MetQuantCred}{3}

\newcommand{\PrincTelec}{Princípios de Telecomunicações III-A}
\newcommand{\PrincTelecSName}{Princípios de Telec. III-A}
\newcommand{\PrincTelecCod}{FEN 05-xxxx}
\newcommand{\PrincTelecCH}{75}
\newcommand{\PrincTelecCred}{4}

\newcommand{\ResMat}{Resistência dos Materiais Básica}
\newcommand{\ResMatSName}{Resistência dos Mat. Básica}
\newcommand{\ResMatCod}{FEN 01-04833}
\newcommand{\ResMatCH}{60}
\newcommand{\ResMatCred}{3}

\newcommand{\SegHig}{Engenharia do Trabalho I}
\newcommand{\SegHigSName}{Engenharia do Trabalho I}
\newcommand{\SegHigCod}{FEN 09-5255}
\newcommand{\SegHigCH}{60}
\newcommand{\SegHigCred}{3}

\newcommand{\AlgComp}{Algoritmos Computacionais I}
\newcommand{\AlgCompSName}{Algoritmos Computac. I}
\newcommand{\AlgCompCod}{FEN 06-xxxx}
\newcommand{\AlgCompCH}{75}
\newcommand{\AlgCompCred}{4}

\newcommand{\AnAlg}{Análise de Algoritmos I}
\newcommand{\AnAlgSName}{Análise de Algoritmos I}
\newcommand{\AnAlgCod}{FEN 06-xxxx}
\newcommand{\AnAlgCH}{60}
\newcommand{\AnAlgCred}{4}

\newcommand{\ArqComp}{Arquitetura de Computadores A}
\newcommand{\ArqCompSName}{Arquitetura de Computadores A}
\newcommand{\ArqCompCod}{FEN 06-xxxx}
\newcommand{\ArqCompCH}{75}
\newcommand{\ArqCompCred}{4}

\newcommand{\Control}{Controle de Processos por Computador I}
\newcommand{\ControlSName}{Controle de Processos por Computador I}
\newcommand{\ControlCod}{FEN 06-xxxx}
\newcommand{\ControlCH}{75}
\newcommand{\ControlCred}{4}

\newcommand{\EngComput}{Engenharia Computacional}
\newcommand{\EngComputSName}{Engenharia Computacional}
\newcommand{\EngComputCod}{FEN 06-xxxx}
\newcommand{\EngComputCH}{75}
\newcommand{\EngComputCred}{4}

\newcommand{\EngCompSoc}{Engenharia de Computação e Sociedade}
\newcommand{\EngCompSocSName}{Engenharia de Computação e Sociedade}
\newcommand{\EngCompSocCod}{FEN 06-xxxx}
\newcommand{\EngCompSocCH}{60}
\newcommand{\EngCompSocCred}{4}

\newcommand{\EngSistA}{Engenharia de Sistemas}
\newcommand{\EngSistASName}{Engenharia de Sistemas}
\newcommand{\EngSistACod}{FEN 06-xxxx}
\newcommand{\EngSistACH}{60}
\newcommand{\EngSistACred}{4}

\newcommand{\ProjBD}{Projeto e Administração de Banco de Dados}
\newcommand{\ProjBDSName}{Projeto e Adm. de Banco de Dados}
\newcommand{\ProjBDCod}{FEN 06-xxxx}
\newcommand{\ProjBDCH}{60}
\newcommand{\ProjBDCred}{4}

\newcommand{\EstSup}{Estágio Supervisionado para Engenharia de Computação}
\newcommand{\EstSupSName}{Estágio Superv. p/ Eng. de Comp.}
\newcommand{\EstSupCod}{FEN 06-xxxx}
\newcommand{\EstSupCH}{300}
\newcommand{\EstSupCred}{10}

\newcommand{\EstrInf}{Estruturas de Informação A}
\newcommand{\EstrInfSName}{Estruturas de Informação A}
\newcommand{\EstrInfCod}{FEN 06-xxxx}
\newcommand{\EstrInfCH}{75}
\newcommand{\EstrInfCred}{4}

\newcommand{\FundComp}{Fundamentos de Computadores}
\newcommand{\FundCompSName}{Fundamentos de Computadores}
\newcommand{\FundCompCod}{FEN 06-xxxx}
\newcommand{\FundCompCH}{90}
\newcommand{\FundCompCred}{5}

\newcommand{\IC}{Inteligência Computacional}
\newcommand{\ICSName}{Inteligência Computacional}
\newcommand{\ICCod}{FEN 06-xxxx}
\newcommand{\ICCH}{60}
\newcommand{\ICCred}{4}

\newcommand{\LabProgA}{Laboratório de Programação A}
\newcommand{\LabProgASName}{Laboratório de Programação A}
\newcommand{\LabProgACod}{FEN 06-xxxx}
\newcommand{\LabProgACH}{75}
\newcommand{\LabProgACred}{4}

\newcommand{\LabProgB}{Laboratório de Programação B}
\newcommand{\LabProgBSName}{Laboratório de Programação B}
\newcommand{\LabProgBCod}{FEN 06-xxxx}
\newcommand{\LabProgBCH}{75}
\newcommand{\LabProgBCred}{4}

\newcommand{\LogProg}{Lógica em Programação}
\newcommand{\LogProgSName}{Lógica em Programação}
\newcommand{\LogProgCod}{FEN 06-xxxx}
\newcommand{\LogProgCH}{60}
\newcommand{\LogProgCred}{4}

\newcommand{\MineraDados}{Mineração de Dados}
\newcommand{\MineraDadosSName}{Mineração de Dados}
\newcommand{\MineraDadosCod}{FEN 06-xxxx}
\newcommand{\MineraDadosCH}{60}
\newcommand{\MineraDadosCred}{4}

\newcommand{\ProcImag}{Processamento Digital de Imagens}
\newcommand{\ProcImagSName}{Processamento Dig. de Imagens}
\newcommand{\ProcImagCod}{FEN 06-xxxx}
\newcommand{\ProcImagCH}{60}
\newcommand{\ProcImagCred}{4}

\newcommand{\CompParal}{Computação Paralela e Distribuída}
\newcommand{\CompParalSName}{Computação Paralela e Distribuída}
\newcommand{\CompParalCod}{FEN 06-xxxx}
\newcommand{\CompParalCH}{75}
\newcommand{\CompParalCred}{4}

\newcommand{\ProjA}{Projeto de Graduação XI-A}
\newcommand{\ProjASName}{Projeto de\\ Graduação XI-A}
\newcommand{\ProjACod}{FEN 06-04578}
\newcommand{\ProjACH}{45}
\newcommand{\ProjACred}{2}

\newcommand{\ProjB}{Projeto de Graduação XI-B}
\newcommand{\ProjBSName}{Projeto de\\ Graduação XI-B}
\newcommand{\ProjBCod}{FEN 06-04635}
\newcommand{\ProjBCH}{45}
\newcommand{\ProjBCred}{2}

\newcommand{\ProjSO}{Projeto de Sistemas Operacionais}
\newcommand{\ProjSOSName}{Projeto de Sistemas Operacionais}
\newcommand{\ProjSOCod}{FEN 06-xxxx}
\newcommand{\ProjSOCH}{75}
\newcommand{\ProjSOCred}{5}

\newcommand{\SistEmb}{Sistemas Embutidos}
\newcommand{\SistEmbSName}{Sistemas Embutidos}
\newcommand{\SistEmbCod}{FEN 06-xxxx}
\newcommand{\SistEmbCH}{60}
\newcommand{\SistEmbCred}{3}

\newcommand{\Telep}{Teleprocessamento e Redes de Computadores I}
\newcommand{\TelepSName}{Teleproc. e Redes de Computadores I}
\newcommand{\TelepCod}{FEN 06-xxxx}
\newcommand{\TelepCH}{60}
\newcommand{\TelepCred}{4}

\newcommand{\TeoComp}{Teoria de Compiladores I}
\newcommand{\TeoCompSName}{Teoria de Compiladores I}
\newcommand{\TeoCompCod}{FEN 06-xxxx}
\newcommand{\TeoCompCH}{75}
\newcommand{\TeoCompCred}{5}

\newcommand{\EletA}{Eletiva Restrita}
\newcommand{\EletASName}{Eletiva Restrita}
\newcommand{\EletACod}{FEN 06-xxxx}
\newcommand{\EletACH}{60}
\newcommand{\EletACred}{3}

\newcommand{\EletB}{Eletiva Restrita}
\newcommand{\EletBSName}{Eletiva Restrita}
\newcommand{\EletBCod}{FEN 06-xxxx}
\newcommand{\EletBCH}{60}
\newcommand{\EletBCred}{3}

\newcommand{\EletC}{Eletiva Restrita}
\newcommand{\EletCSName}{Eletiva Restrita}
\newcommand{\EletCCod}{FEN 06-xxxx}
\newcommand{\EletCCH}{60}
\newcommand{\EletCCred}{3}

\newcommand{\EletRec}{Reconhecimento de Padrões}
\newcommand{\EletRecSName}{Reconhecimento de Padrões}
\newcommand{\EletRecCod}{FEN 06-xxxx}
\newcommand{\EletRecCH}{60}
\newcommand{\EletRecCred}{3}

\newcommand{\EletRedes}{Redes de Interconexão}
\newcommand{\EletRedesSName}{Redes de Interconexão}
\newcommand{\EletRedesCod}{FEN 06-xxxx}
\newcommand{\EletRedesCH}{60}
\newcommand{\EletRedesCred}{3}

\newcommand{\EletGeo}{Geomática}
\newcommand{\EletGeoSName}{Geomática}
\newcommand{\EletGeoCod}{FEN 06-xxxx}
\newcommand{\EletGeoCH}{60}
\newcommand{\EletGeoCred}{3}

\newcommand{\EletArq}{Arquiteturas Avançadas de Computadores}
\newcommand{\EletArqSName}{Arquiteturas Avançadas de Computadores}
\newcommand{\EletArqCod}{FEN 06-xxxx}
\newcommand{\EletArqCH}{60}
\newcommand{\EletArqCred}{3}

\newcommand{\EletMov}{Programação para Dispositivos Móveis}
\newcommand{\EletMovSName}{Programação para Dispositivos Móveis}
\newcommand{\EletMovCod}{FEN 06-xxxx}
\newcommand{\EletMovCH}{60}
\newcommand{\EletMovCred}{3}

\newcommand{\EletPadroes}{Padrões de Projetos Orientados a Objetos}
\newcommand{\EletPadroesSName}{Padrões de Projetos Orientados a Objetos}
\newcommand{\EletPadroesCod}{FEN 06-xxxx}
\newcommand{\EletPadroesCH}{60}
\newcommand{\EletPadroesCred}{3}


 %%%%%%%%%%%%%%%%%%%%%%%%%%%%%%%%%%%%%%%%%%%%%%%%%%%%%%%%%%%%%%%%%%%%
%Básico
\normalsize 

\begin{document}

\begin{textblock*}{3cm}[0.5,0.5](4cm,4cm)
    \includegraphics[width=3cm]{imagens/logo_uerj_cor.jpg}
\end{textblock*}
\begin{textblock*}{12cm}(6cm,3cm)   % 6.375=8.5 - 1.5 - 0.625
    Universidade do Estado do Rio de Janeiro\\
    Sub-Reitoria de Graduação \\
    Faculdade de Engenharia\\
    Departamento de Engenharia de Sistemas e Computação
\end{textblock*}



\title{Projeto Pedagógico do Curso de Engenharia de Computação}
\subtitle{2016}
\date{}
\maketitle


\begin{minipage}{\textwidth}
\vfill
\tableofcontents
\end{minipage}
\frontmatter%%%%%%%%%%%%%%%%%%%%%%%%%%%%%%%%%%%%%%%%%%%%%%%%%%%%%%
%\include{dedic}
%\include{pref} 

%\tableofcontents


\mainmatter%%%%%%%%%%%%%%%%%%%%%%%%%%%%%%%%%%%%%%%%%%%%%%%%%%%%%%%
%\include{part}
\include{Capitulos}
%\include{chapter}
\backmatter%%%%%%%%%%%%%%%%%%%%%%%%%%%%%%%%%%%%%%%%%%%%%%%%%%%%%%%

\appendix
\appendixpage
\chapter{Deliberação nº 33/95 da UERJ}
\label{delib3395}
%\includepdf[pages=-,pagecommand={\thispagestyle{plain}}]{pdf/Deliberacao33-95.pdf}
\includepdf[pages=-,pagecommand={\thispagestyle{fancy}}]{leis/Deliberacao33-95.pdf}

\chapter{Resolução CNE/CES nº 11}
\label{cne11}
\includepdf[pages=-,pagecommand={\thispagestyle{fancy}}]{leis/CES112002.pdf}
%\includepdf[pages=-]{leis/CES112002.pdf}

\chapter{Resolução nº 1.010 CREA/CONFEA}
\label{res1010}
\includepdf[pages=1,pagecommand={\thispagestyle{fancy}}]{leis/res1010.pdf}
\includepdf[pages=2, offset=1.8cm 0,pagecommand={\thispagestyle{fancy}}]{leis/res1010.pdf}
\includepdf[pages=3,pagecommand={\thispagestyle{fancy}}]{leis/res1010.pdf}
\includepdf[pages=4, offset=1.8cm 0,pagecommand={\thispagestyle{fancy}}]{leis/res1010.pdf}
\includepdf[pages=5,pagecommand={\thispagestyle{fancy}}]{leis/res1010.pdf}
\includepdf[pages=6, offset=1.8cm 0,pagecommand={\thispagestyle{fancy}}]{leis/res1010.pdf}
\includepdf[pages=7,pagecommand={\thispagestyle{fancy}}]{leis/res1010.pdf}

\chapter{Fluxograma do Curso de Engenharia de Computação}
\label{fluxograma}
\includepdf[pages=-,angle=90]{pdf/fluxogramaEngenhariaComputacao.pdf}
\chapter{Ementas do Curso de Engenharia de Computação}
\label{ementas}
\includepdf[pages=-,addtotoc={1,section,1,{\Adm},},pagecommand={\thispagestyle{fancy}}]{ementasExternas/Administracao.pdf}
\includepdf[pages=-,addtotoc={1,section,1,{\AlgLin},},pagecommand={\thispagestyle{fancy}}]{ementasExternas/AlgebraLinearIII.pdf}
\includepdf[pages=-,addtotoc={1,section,1,{\AlgComp},},pagecommand={\thispagestyle{fancy}}]{pdf/AlgoritmosComputacionais.pdf}
\includepdf[pages=-,addtotoc={1,section,1,{\AnAlg},},pagecommand={\thispagestyle{fancy}}]{pdf/AnaliseDeAlgoritmos.pdf}
\includepdf[pages=-,addtotoc={1,section,1,{\AnaFis},},pagecommand={\thispagestyle{fancy}}]{pdf/AnaliseDeSistemasFisicos.pdf}

\includepdf[pages=-,addtotoc={1,section,1,{\AnaVet},},pagecommand={\thispagestyle{fancy}}]{ementasExternas/AnaliseVetorial.pdf}
\includepdf[pages=-,addtotoc={1,section,1,{\EletArq},},pagecommand={\thispagestyle{fancy}}]{pdf/Eletiva4_ComputacaoDeAltoDesempenho.pdf}
\includepdf[pages=-,addtotoc={1,section,1,{\ArqComp},},pagecommand={\thispagestyle{fancy}}]{pdf/ArquiteturaDeComputadores.pdf}
\includepdf[pages=-,addtotoc={1,section,1,{\CalcI},},pagecommand={\thispagestyle{fancy}}]{ementasExternas/CalculoINew.pdf}
\includepdf[pages=-,addtotoc={1,section,1,{\CalcII},},pagecommand={\thispagestyle{fancy}}]{ementasExternas/CalculoIINew.pdf}

\includepdf[pages=-,addtotoc={1,section,1,{\CEV},},pagecommand={\thispagestyle{fancy}}]{pdf/CircuitosEletricosV.pdf}
\includepdf[pages=-,addtotoc={1,section,1,{\CEVI},},pagecommand={\thispagestyle{fancy}}]{ementasExternas/CircuitosEletricosIV.pdf}
\includepdf[pages=-,addtotoc={1,section,1,{\CompParal},},pagecommand={\thispagestyle{fancy}}]{pdf/ComputacaoParalela.pdf}
\includepdf[pages=-,addtotoc={1,section,1,{\Control},},pagecommand={\thispagestyle{fancy}}]{pdf/ControleDeProcessosPorComputador.pdf}

\includepdf[pages=-,addtotoc={1,section,1,{\CServMec},},pagecommand={\thispagestyle{fancy}}]{ementasExternas/ControleEServomecanismosIII.pdf}
\includepdf[pages=-,addtotoc={1,section,1,{\DesBas},},pagecommand={\thispagestyle{fancy}}]{ementasExternas/DesenhoBasico.pdf}
\includepdf[pages=-,addtotoc={1,section,1,{\EletI},},pagecommand={\thispagestyle{fancy}}]{ementasExternas/EletronicaI.pdf}
\includepdf[pages=-,addtotoc={1,section,1,{\EletIIA},},pagecommand={\thispagestyle{fancy}}]{ementasExternas/EletronicaII.pdf}
\includepdf[pages=-,addtotoc={1,section,1,{\EngComput},},pagecommand={\thispagestyle{fancy}}]{pdf/EngenhariaComputacional.pdf}

\includepdf[pages=-,addtotoc={1,section,1,{\EngSistA},},pagecommand={\thispagestyle{fancy}}]{pdf/EngenhariaDeSistemas.pdf}
\includepdf[pages=-,addtotoc={1,section,1,{\EngCompSoc},},pagecommand={\thispagestyle{fancy}}]{pdf/EngenhariaDeComputacaoESociedade.pdf}
\includepdf[pages=-,addtotoc={1,section,1,{\SegHig},},pagecommand={\thispagestyle{fancy}}]{pdf/EngenhariaDoTrabalhoI.pdf}
\includepdf[pages=-,addtotoc={1,section,1,{\CalcIII},},pagecommand={\thispagestyle{fancy}}]{ementasExternas/CalculoIIINew.pdf}
\includepdf[pages=-,addtotoc={1,section,1,{\EstSup},},pagecommand={\thispagestyle{fancy}}]{pdf/EstagioSupervisionadoXIA.pdf}
\includepdf[pages=-,addtotoc={1,section,1,{\EstrInf},},pagecommand={\thispagestyle{fancy}}]{pdf/EstruturasDeInformacao.pdf}

\includepdf[pages=-,addtotoc={1,section,1,{\FenTran},},pagecommand={\thispagestyle{fancy}}]{ementasExternas/FenomenosDeTransporte.pdf}
\includepdf[pages=-,addtotoc={1,section,1,{\FisI},},pagecommand={\thispagestyle{fancy}}]{ementasExternas/FisicaI.pdf}
\includepdf[pages=-,addtotoc={1,section,1,{\FisII},},pagecommand={\thispagestyle{fancy}}]{ementasExternas/FisicaII.pdf}
\includepdf[pages=-,addtotoc={1,section,1,{\FisIII},},pagecommand={\thispagestyle{fancy}}]{ementasExternas/FisicaIII.pdf}
\includepdf[pages=-,addtotoc={1,section,1,{\FisIV},},pagecommand={\thispagestyle{fancy}}]{ementasExternas/FisicaIV.pdf}

\includepdf[pages=-,addtotoc={1,section,1,{\FundComp},},pagecommand={\thispagestyle{fancy}}]{pdf/FundamentosDeComputadores.pdf}
\includepdf[pages=-,addtotoc={1,section,1,{\EletGeo},},pagecommand={\thispagestyle{fancy}}]{pdf/Eletiva3_Geomatica.pdf}
\includepdf[pages=-,addtotoc={1,section,1,{\GeoAna},},pagecommand={\thispagestyle{fancy}}]{ementasExternas/GeometriaAnalitica.pdf}
\includepdf[pages=-,addtotoc={1,section,1,{\GDSName},},pagecommand={\thispagestyle{fancy}}]{ementasExternas/GeometriaDescritivaI.pdf}
\includepdf[pages=-,addtotoc={1,section,1,{\IC},},pagecommand={\thispagestyle{fancy}}]{pdf/InteligenciaComputacional.pdf}

\includepdf[pages=-,addtotoc={1,section,1,{\IntEco},},pagecommand={\thispagestyle{fancy}}]{ementasExternas/IntroducaoAEconomia.pdf}
\includepdf[pages=-,addtotoc={1,section,1,{\IntAmb},},pagecommand={\thispagestyle{fancy}}]{ementasExternas/IntroducaoAEngenhariaAmbiental.pdf}
\includepdf[pages=-,addtotoc={1,section,1,{\LabProgA},},pagecommand={\thispagestyle{fancy}}]{pdf/LaboratorioDeProgramacaoA.pdf}
\includepdf[pages=-,addtotoc={1,section,1,{\LabProgB},},pagecommand={\thispagestyle{fancy}}]{pdf/LaboratorioDeProgramacaoB.pdf}
\includepdf[pages=-,addtotoc={1,section,1,{\LogProg},},pagecommand={\thispagestyle{fancy}}]{pdf/LogicaEmProgramacao.pdf}

\includepdf[pages=-,addtotoc={1,section,1,{\MatEle},},pagecommand={\thispagestyle{fancy}}]{pdf/MateriaisEletricosEMagneticos.pdf}
\includepdf[pages=-,addtotoc={1,section,1,{\MecTec},},pagecommand={\thispagestyle{fancy}}]{ementasExternas/MecanicaTecnica.pdf}
\includepdf[pages=-,addtotoc={1,section,1,{\MetQuant},},pagecommand={\thispagestyle{fancy}}]{pdf/MetodosQuantitativos.pdf}
\includepdf[pages=-,addtotoc={1,section,1,{\MineraDados},},pagecommand={\thispagestyle{fancy}}]{pdf/MineracaoDeDados.pdf}
\includepdf[pages=-,addtotoc={1,section,1,{\ModMat},},pagecommand={\thispagestyle{fancy}}]{ementasExternas/ModelosMatematicos.pdf}

\includepdf[pages=-,addtotoc={1,section,1,{\EletPadroes},},pagecommand={\thispagestyle{fancy}}]{pdf/Eletiva6_Padroes.pdf}
\includepdf[pages=-,addtotoc={1,section,1,{\PrincTelec},},pagecommand={\thispagestyle{fancy}}]{ementasExternas/PrincipiosDeTelecomunicacoesIII.pdf}
\includepdf[pages=-,addtotoc={1,section,1,{\ProbEst},},pagecommand={\thispagestyle{fancy}}]{ementasExternas/ProbEst.pdf}
\includepdf[pages=-,addtotoc={1,section,1,{\ProcImag},},pagecommand={\thispagestyle{fancy}}]{pdf/ProcessamentoDeImagens.pdf}
\includepdf[pages=-,addtotoc={1,section,1,{\EletMov},},pagecommand={\thispagestyle{fancy}}]{pdf/Eletiva5_ProgramacaoParaDispositivosMoveis.pdf}

\includepdf[pages=-,addtotoc={1,section,1,{\ProjSO},},pagecommand={\thispagestyle{fancy}}]{pdf/ProjetoDeSistemasOperacionais.pdf}
\includepdf[pages=-,addtotoc={1,section,1,{\ProjA},},pagecommand={\thispagestyle{fancy}}]{pdf/ProjetoXIA.pdf}
\includepdf[pages=-,addtotoc={1,section,1,{\ProjB},},pagecommand={\thispagestyle{fancy}}]{pdf/ProjetoDeGraduacaoB.pdf}
\includepdf[pages=-,addtotoc={1,section,1,{\ProjBD},},pagecommand={\thispagestyle{fancy}}]{pdf/ProjetoEAdministracaoDeBancoDeDados.pdf}
\includepdf[pages=-,addtotoc={1,section,1,{\QuiX},},pagecommand={\thispagestyle{fancy}}]{ementasExternas/QuimicaX.pdf}

\includepdf[pages=-,addtotoc={1,section,1,{\EletRec},},pagecommand={\thispagestyle{fancy}}]{pdf/Eletiva1_ReconhecimentoDePadroes.pdf}
\includepdf[pages=-,addtotoc={1,section,1,{\EletRedes},},pagecommand={\thispagestyle{fancy}}]{pdf/Eletiva2_RedesDeInterconexao.pdf}
\includepdf[pages=-,addtotoc={1,section,1,{\ResMat},},pagecommand={\thispagestyle{fancy}}]{ementasExternas/ResMat.pdf}
\includepdf[pages=-,addtotoc={1,section,1,{\SistEmb},},pagecommand={\thispagestyle{fancy}}]{pdf/SistemasEmbutidos.pdf}
\includepdf[pages=-,addtotoc={1,section,1,{\Telep},},pagecommand={\thispagestyle{fancy}}]{pdf/TeleprocessamentoERedes.pdf}

\includepdf[pages=-,addtotoc={1,section,1,{\TeoComp},},pagecommand={\thispagestyle{fancy}}]{pdf/TeoriaDeCompiladores.pdf}


\addappheadtotoc


%\printindex

%%%%%%%%%%%%%%%%%%%%%%%%%%%%%%%%%%%%%%%%%%%%%%%%%%%%%%%%%%%%%%%%%%%%%%

\end{document}





